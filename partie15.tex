\section{Introduction à la programmation logique}
\label{p2}
\subsection{Introduction à la programmation logique}

Prolog est l’un des principaux langages de programmation logique. Il est à la base de beaucoup de chose. 

La programmation logique fait de la déduction sur les axiomes.
On utilise la logique comme un langage de programmation : on va adapter l’algorithme de réfutation qu’on a vu précédemment

Le programme (ressemble à une théorie):
\begin{itemize}
	\item Axiomes en logique des prédicats
	\item Une requête, un but (=goal) -> le but du système est de prouver quelque chose
	\item Un prouveur de théorème -> attention: il faut des conditions sur le prouveur car on doit être capable de prévoir le temps et l’espace utilisé par le programme.
\end{itemize}

\paragraph{}
Exécuter un programme = faire des déductions en essayant de prouver le but. Mais est-ce que cette idée peut donner un système de programmation pratique?

\paragraph{}
Il y a une tension entre expressivité et efficacité: Si c’est trop expressif, ça devient moins efficace, par contre si c’est trop peu expressif, on ne peut rien programmer, ça ne sert à rien non plus. Il faut donc être expressif tout en restant efficace. Le Prolog offre un bon mélange entre expressivité et efficacité.

\paragraph{}
Mais pour arriver à cela, il y a quelques problèmes à surmonter:
\begin{enumerate}
	\item un prouveur est limité
		vérité = $ p \models q $ (= $ q $ est vrai dans tous les modèles de $ p $)\\
		preuve = $ p  \vdash q $ \\
		$ p \models q \Rightarrow p \vdash q $ (= Si c’est vrai dans tous les modèles, on peut trouver une preuve)\\
		Si $ p \models q $ alors l’algorithme se terminera. Cependant, on ne peut pas trouver les preuves pour des choses vraies dans tous les modèles. (Comme c’est impossible on ne prend qu’une partie des modèles. Ceci est une limitation du programme).
		
		\item b) Même si on peut trouver une preuve, le prouveur est peut-être inefficace (utilise trop de temps ou de mémoire) ou imprévisible. $ \to $ On ne peut pas raisonner sur l’efficacité du prouveur.
		
		\item c) La déduction faite par le prouveur doit être constructive  
\end{enumerate}

		Si le prouveur dit: $ (\exists X) P(X) $  alors le prouveur doit donner une valeur de x (c’est quoi $ x $).


